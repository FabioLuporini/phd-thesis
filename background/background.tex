\chapter{Background}

\section{The Finite Element Method}
\label{sec:bkg:fem}
Computational methods based upon finite elements are used to approximate the solution of partial differential equations (henceforth, PDEs) in a wide variety of domains. The mathematical abstraction used in finite element methods (or FEMs) is extremely powerful: not only does it help reasoning about the problem, but also provides systematic ways of deriving effective computer implementations. In~\cite{brenner-and-scott}, it is usefully suggested to consider an FEM as a black box that, given a differential equation, returns a discrete algorithm capable of approximating the equation solution. Unveiling the whole magic of such a black box is clearly out of the scope of this chapter. We rather limit ourselves to review the mathematical properties and the computational aspects that are essential for understanding the contributions in Chapters~\ref{ch:optimality} and~\ref{ch:lowlevelopt}. The content and the notation used in this section are inspired from~\cite{florian-thesis,kirby-and-logg,olgaard-and-wells}. For a complete treatment of the subject, the reader is invited to refer to~\cite{brenner-and-scott}.



\subsection{Variational Formulation}
\label{sec:bkg:var-problems}
We consider the weak formulation of a {\em linear variational problem}

\begin{equation}
\begin{split}
\text{Find}\ u \in U\ \text{such that} \\
a(u, v) = L(v), \forall v \in V
\end{split}
\end{equation}

where $a$ and $L$ are, respectively, a bilinear form and a linear form. The term ``variational'' stems from the fact that the function $v$ can vary arbitrarily. The unfamiliar reader may find this expression unusual. Understanding the actual meaning of this formulation is far beyond the goals of this review, since an entire course in functional analysis would be needed. Informally, we can think of $V$ as a ``very nice'' space, in which functions have desirable properties. The underlying idea of the variational formulation is then to ``transfer'' certain requirements (e.g., differentiability) on the unknown $u$ to $v$.

The sets $U$ and $V$ and called, respectively, trial and test functions. The variational problem is discretized by using discrete test and trial spaces

\begin{equation}
\label{sec:bkg:eq:disc-var}
\begin{split}
\text{Find}\ u_h \in U_h \subset U\ \text{such that} \\
a(u_h, v_h) = L(v_h), \forall v_h \in V_h \subset V
\end{split}
\end{equation}

Let $\lbrace \psi_i \rbrace_{i=1}^N$ be the set of basis functions spanning $U_h$ and let $\lbrace \phi_j \rbrace_{j=1}^{N}$ be the set of basis functions spanning $V_h$. Then the unknown solution $u$ can be approximated as a linear combination of the basis functions $\lbrace \psi_i \rbrace_{i=1}^N$,

\begin{equation}
u_h = \sum_{j=1}^N U_j \psi_j.
\end{equation}

This allows us to rewrite~\ref{sec:bkg:eq:disc-var} as:

\begin{equation}
\sum_{j=1}^N U_j a(\psi_j, \phi_i) = L(\phi_i),\ i=1,2,...,N
\end{equation}

From the solution of the following linear system we determine the set of {\em degrees of freedom} $U$ to express $u_h$:

\begin{equation}
Au = b
\end{equation}

where clearly

\begin{equation}
\label{sec:bkg:eq:disc-op}
\centering
\begin{split}
A_{ij} = a(\phi_i(x), \phi_j(x)) \\
b_i = L(\phi_i(x))
\end{split}
\end{equation}

The matrix $A$ and the vector $b$ can be seen as the discrete operators arising from the bilinear form $a$ and from the linear form $L$ for the given choice of basis functions.

The variational formulation of a {\em non-linear variational problem} requires refinements that are out of the scope of this review. The interested reader should refer to~\cite{brenner-and-scott}.

\subsection{Finite Elements}
In an FEM the domain $\Omega$ of the PDE is partitioned into a finite set of disjoint cells $\lbrace K \rbrace$; that is, $\bigcup K = \Omega$ and $\bigcap K = \emptyset$. This forms a {\em mesh}. A finite element is usually defined (see for example~\cite{brenner-and-scott}) as a triple ${<}K,\mathcal{P}_K,\mathcal{L}_K{>}$, where:
\begin{itemize}
\item $K$ is a cell in the mesh with non-empty interior and piecewise smooth boundary;
\item $\mathcal{P}_K$ is a finite dimensional ``local'' function space of dimension $n_K$;
\item the set of degrees of freedom $\mathcal{L}_K$ is a basis $\lbrace l_1^K, l_2^K, ..., l_{n_K}^K\rbrace$ for $\mathcal{P}'_K$, the dual space of $\mathcal{P}_K$. 
\end{itemize}
This definition allows to impose constraints on the set of basis functions $\lbrace \phi_1^K, \phi_2^K, ..., \phi_{n_K}^K\rbrace$ spanning $\mathcal{P}_K$. For instance, to enforce a {\em nodal basis} for $\mathcal{P}_K$ -- a particularly useful property for expressing solutions in $U_h$ -- we can impose that the relationship

\begin{equation}
l_i^K(\phi_j^K) = \delta_{ij},\ i,j = 1,2,...,n_K
\end{equation}

must be satisfied ($\delta_{ij}$ is the Kronecker delta). This allows the express any $v \in \mathcal{P}_K$ as

\begin{equation}
v = \sum_{i=1}^{n_K} l_i^K(v) \phi_i^K.
\end{equation}

Each linear functional in $\mathcal{L}_K$ is used to evaluate one degree of freedom of $v$ in terms of the chosen nodal basis. In other words, we can refer to both the coefficients $U$ introduced in the previous section and $\mathcal{L}_K$ as the degrees of freedom.

\paragraph{Example: the triangular Lagrange element}
As an example\footnote{The example is extracted from~\cite{fenics-book}}, consider a triangular cell $K$ and let $\mathcal{P}_K$ be the space of polynomials of order $1$ on $K$. $\mathcal{L}_K$ may be the set of bounded linear functionals representing point evaluation at the vertices $\boldsymbol{x}^i$ ($i=1,2,3$) of $K$ such that

\begin{equation}
\centering
\begin{split}
l_i^K : \mathcal{P^K} \rightarrow \mathbb{R} \\
l_i^K(v) = v(\boldsymbol{x}^i)
\end{split}
\end{equation}

Since if $v$ is zero at each vertex then $v$ must be zero everywhere, $\mathcal{L}_K$ really is a basis for $\mathcal{P}'_K$, so what we have defined is indeed a finite element. In particular, if we take $\boldsymbol{x}^1 = (0, 0),\ \boldsymbol{x}^2 = (1,0),\ \boldsymbol{x}^3 = (0,1)$, we have that the nodal basis is given by:

\begin{equation}
\phi_1(\boldsymbol{x}) = 1 - x_1 - x_2,\ \ \phi_2(\boldsymbol{x}) = x_1,\ \ \phi_3(\boldsymbol{x}) = x_2.
\end{equation}


\subsection{Global Discretization}
A {\em local-to-global mapping} allows to patch together the finite elements to form a global function space, for instance the set of trial functions $U_h = \operatorname{span}\lbrace \psi_i \rbrace_{i=1}^N$ introduced in Section~\ref{sec:bkg:var-problems}. A local-to-global mapping is a function

\begin{equation}
\iota_K : [1,n_K] \rightarrow [1,N]
\end{equation}

that maps the local degrees of freedom $\mathcal{L}_K$ to global degrees of freedom $\mathcal{L}$. The mappings $\iota_K$, together with the choice of $\mathcal{L}_K$, determine the continuity of a function space or, in simpler words, the continuity of a function throughout the domain $\Omega$. The reader is invited to refer to~\cite{fenics-book} for a comprehensive yet simple description of this step.

One of the crucial aspects of an FEM is that global function spaces are often defined in terms of a {\em reference finite element} ${<}\hat{K}, \hat{\mathcal{P}}, \hat{\mathcal{L}}{>}$ and a set of invertible mappings $\lbrace \mathcal{G}_K\rbrace_{K}$ from $\hat{K}$ to each cell in the mesh such that $K = \mathcal{G}_K(\hat{K})$.  This situation is illustrated in Figure~\ref{fig:bkg:reference-el}.

For each $K$, $\mathcal{G}_K$ also allows to generate $\mathcal{P}_K$ and $\mathcal{L}_K$.  The complexity of this process depends on the mapping itself. In the simplest case, the mapping is affine; that is, expressible as $\mathcal{G}_K(\hat{\boldsymbol{x}}) = A_K \hat{\boldsymbol{x}} + b_K$, where $A_K$ and $b_K$ are, respectively, some matrix and vector.


\subsection{Assembly}
\label{sec:bkg:assembly}
The {\em assembly} of an FEM is the phase in which the matrix $A$ and the vector $b$ from~\ref{sec:bkg:eq:disc-op} are constructed. This is accomplished by adding the contributions from each $K$ to $A$ and $b$. Let us consider the bilinear form $a$, and assume this is an integral over $\Omega$. Thanks to the linearity of the operator, we can express $a$ as

\begin{equation}
a = \sum_{K} a_K
\end{equation}

where $a_K$ is an element bilinear form. We can then define the local element matrix

\begin{equation}
A_i^K = a_K (\psi_{i_1}^K, \phi_{i_2}^K)
\end{equation}

where $i \in \mathcal{I}_K$ is the index set on $A_i^K$. That is, $\mathcal{I}_K = \lbrace (1,1), (1,2), ..., (n_U, n_V) \rbrace$, with $n_U$ and $n_V$ representing the number of degrees of freedom for the local trial functions $\psi^K \in U_h^K$ and the local test functions $\phi^K \in V_h^K$. The element matrix $A^K$ is therefore a (typically dense) matrix of dimension $n_U \times n_V$.

Now let $\iota_K^U$ and $\iota_K^V$ be the local-to-global mappings for the local discrete function spaces $U_h^K$ and $V_h^K$. We can define, for each $K$, the collective local-to-global mapping $\iota_K : \mathcal{I}_K \rightarrow \mathcal{I}$ such that

\begin{equation}
\iota_K (i) = (\iota_K^U(i_1), \iota_K^V(i_2))\ \forall i \in \mathcal{I}_K.
\end{equation}

This simply maps a pair of local degrees of freedom to a pair of global degrees of freedom. Let also $\mathcal{T}$ be the subset of the cells in the mesh in which $\psi_{i_1}$ and $\phi_{i_2}$ are both non-zero. Note that here we are talking about the global functions whose restrictions to $K$ gives $\psi_{i_1}^K$ and $\phi_{i_2}^K$. By construction, $\iota_K$ is invertible if $K \in \mathcal{T}$. At this point, we have all the ingredients to compute $A$ as the sum of local contributions from the cells in the mesh:

\begin{equation}
\begin{split}
A_i & = \sum_{K \in \mathcal{T}} a_K (\psi_{i_1},\phi_{i_2}) = \\
& = \sum_{K \in \mathcal{T}} a_K(\psi_{(\iota_K^U)^{-1}(i_1)}^K, \phi_{(\iota_K^V)^{-1}(i_2)}^K) = \sum_{K \in \mathcal{T} A_{\iota_K^{-1}(i)}^K}
\end{split}
\end{equation} 

Similar conclusions may be drawn for the linear form $L$. We observe that this computation can be implemented as a single iteration over all cells in the mesh. On each cell, $A^K$ is computed and added to $A$ using the inverse mappings. This approach is particularly efficient because it only evaluates the non-zero entries in the sparse matrix $A$. Other more trivial implementations of the assembly phase are possible, although rarely used in practice because ineffective. 

We conclude with a useful clarification about the terminology. The assembly process is often naturally characterized as a two-step procedure: {\em local assembly}  and {\em global assembly}. The former consists of computing the contributions of each single element (i.e., the element matrices $A^K$); the latter represents the ``coupling'' of all $A^K$ into $A$. As we shall see, one of the main subjects of this thesis is the computational optimization of the local assembly phase.

\subsection{Local Assembly: from Math to Code}
%
%We illustrate local assembly in a concrete example, the evaluation of the local element matrix for a Laplacian operator. Consider the weighted Laplace equation
%\begin{equation}
%- \nabla \cdot (w \nabla u) = 0
%\end{equation}
%in which $u$ is unknown, while $w$ is prescribed. The bilinear form associated with the weak variational form of the equation is:
%\begin{equation}
%a(v, u) = \int_\Omega w \nabla v \cdot \nabla u\ \mathrm{d}x
%\end{equation}
%The domain $\Omega$ of the equation is partitioned into a set of cells (elements) $T$ such that $\bigcup T = \Omega$ and $\bigcap T = \emptyset$. By defining $\lbrace \phi_i^K \rbrace$ as the set of local basis functions spanning $U$ on the element $K$, we can express the local element matrix as
%\begin{equation}
%\label{stiffness}
%A_{ij}^K = \int_K w \nabla \phi_i^K \cdot \nabla \phi_j^K\ \mathrm{d}x
%\end{equation}
%The local element vector $L$ can be determined in an analogous way. 
%
%\subsection{Monomials}
%\label{sec:monomials}
%It has been shown (e.g., in~\citeN{Kirby:TC}) that local element tensors can be expressed as a sum of integrals over $K$, each integral being the product of derivatives of functions from sets of discrete spaces and, possibly, functions of some spatially varying coefficients. An integral of this form is called \textit{monomial}. 
%
%%From the computational perspective, its evaluation is however less expensive than that of $A$.


%\subsection{Quadrature mode}
%Quadrature schemes are typically used to numerically evaluate $A_{ij}^K$. For convenience, a reference element $K_0$ and an affine mapping $F_K : K_0 \rightarrow K$ to any element $K \in T$ are introduced. This implies that a change of variables from reference coordinates $X_0$ to real coordinates $x = F_K (X_0)$ is necessary any time a new element is evaluated. The numerical integration routine based on quadrature over an element $K$ can be expressed as follows
%\begin{equation}
%\label{eq:quadrature}
%A_{ij}^K = \sum_{q=1}^N \sum_{\alpha_3=1}^n \phi_{\alpha_3}(X^q)w_{\alpha_3} \sum_{\alpha_1=1}^d \sum_{\alpha_2=1}^d \sum_{\beta=1}^d \frac{\partial X_{\alpha_1}}{\partial x_{\beta}} \frac{\partial \phi_i^K(X^q)}{\partial X_{\alpha_1}} \frac{\partial X_{\alpha_2}}{\partial x_{\beta}} \frac{\partial \phi_j^K(X^q)}{\partial X_{\alpha_2}} det F_K' W^q
%\end{equation}
%where $N$ is the number of integration points, $W^q$ the quadrature weight at the integration point $X^q$, $d$ is the dimension of $\Omega$, $n$ the number of degrees of freedom associated to the local basis functions, and $det$ the determinant of the Jacobian matrix used for the aforementioned change of coordinates.  
%
%% Magari la summation sui coefficients la sputo dentro? che dici?
%
%\subsection{Tensor contraction mode}
%\label{sec:tc}
%Starting from Equation~\ref{eq:quadrature}, exploiting linearity, associativity and distributivity of the involved mathematical operators, we can rewrite the expression as
%\begin{equation}
%\label{eq:tensor}
%A_{ij}^K = \sum_{\alpha_1=1}^d \sum_{\alpha_2=1}^d \sum_{\alpha_3=1}^n det F_K' w_{\alpha_3} \sum_{\beta=1}^d \frac{X_{\alpha_1}}{\partial x_{\beta}} \frac{\partial X_{\alpha_2}}{\partial x_{\beta}} \int_{K_0} \phi_{\alpha_3} \frac{\partial \phi_{i_1}}{\partial X_{\alpha_1}} \frac{\partial \phi_{i_2}}{\partial X_{\alpha_2}} dX.
%\end{equation}
%A generalization of this transformation has been proposed in~\cite{Kirby:TC}. Because of only involving reference element terms, the integral in the equation can be pre-evaluated and stored in temporary variables. The evaluation of the local tensor can then be abstracted as
%\begin{equation}
%A_{ij}^K = \sum_{\alpha} A_{i_1 i_2 \alpha}^0 G_{K}^\alpha
%\end{equation}
%in which the pre-evaluated ``reference tensor'' $A_{i_1 i_2 \alpha}$ and the cell-dependent ``geometry tensor'' $G_{K}^\alpha$ are exposed. 
%
%\subsection{Qualitative comparison}
%\label{sec:qualitative}
%Depending on form and discretization, the relative performance of the two modes, in terms of the operation count, can vary quite dramatically. The presence of derivatives or coefficient functions in the input form increases the rank of the geometry tensor, making the traditional quadrature mode preferable for ``complex'' forms. On the other hand, speed-ups from adopting tensor mode can be significant in a wide class of forms in which the geometry tensor remains ``sufficiently small''. The discretization, particularly the relative polynomial order of trial, test, and coefficient functions, also plays a key role in the resulting operation count. 
%
%These two modes have been implemented in the FEniCS Form Compiler (\cite{FFC-TC}). In this compiler, a heuristic is used to choose the most suitable mode for a given form. It consists of analysing each monomial in the form, counting the number of derivatives and coefficient functions, and checking if this number is greater than a constant found empirically (\cite{Fenics}). We will later comment on the efficacy of this approach (Section~\ref{sec:perf-results}). For the moment, we just recall that one of the goals of this research is to produce a system that goes beyond the dichotomy between quadrature and tensor modes. We will reason in terms of loop nests, code motion, and code pre-evaluation, searching the entire implementation space for an optimal synthesis.  



\subsubsection{Code Examples}
\label{sec:bkg:mathcode}
We have explained that local assembly is the computation of contributions of a specific cell in the discretized domain to the linear system which yields the PDE solution. The process consists of numerically evaluating problem-specific integrals to produce a matrix and a vector (only the derivation of the matrix was shown in Section~\ref{sec:quadrature-rep}), whose sizes depend on the order of the method. This operation is applied to all cells in the discretized domain (mesh).

We consider again the weighted Laplace example of the previous section. A C-code implementation of Equation~\ref{eq:quadrature} is illustrated in Listing~\ref{code:weighted-laplace}. The values at the various quadrature points of basis functions ($\phi$) derivatives are tabulated in the A and B arrays. The summation along quadrature points $q$ is implemented by the \emph{i} loop, whereas the one along $\alpha_3$ is represented by the \emph{r} loop. In this example, we assume $d=2$ (2D mesh), so the summations along $\alpha_1$, $\alpha_2$ and $\beta$ have been straightforwardly expanded in the expression that evaluates the local element matrix $A$. 

\begin{algorithm}[t]
\scriptsize\ttfamily
\SetAlgorithmName{LISTING}{}

\KwSty{void} weighted$\_$laplace(\KwSty{double} A[3][3], \KwSty{double} **coords, \KwSty{double} w[3]) \\
$\lbrace$ \\
~~// Compute Jacobian \\
~~\KwSty{double} J[4]; \\
~~compute$\_$jacobian$\_$triangle$\_$2d(J, coords); \\
~~\\
~~// Compute Jacobian inverse and determinant \\
~~\KwSty{double} K[4]; \\
~~\KwSty{double} detJ; \\
~~compute$\_$jacobian$\_$inverse$\_$triangle$\_$2d(K, detJ, J); \\
~~\KwSty{const double} det = fabs(detJ); \\
~~\\
~~// Quadrature weights \\
~~\KwSty{static const double} W[6] = {0.5}; \\
~~\\
~~// Basis functions \\
~~\KwSty{static const double} B[6][3] = $\lbrace\lbrace$...$\rbrace\rbrace$ ;\\
~~\KwSty{static const double} C[6][3] = $\lbrace\lbrace$...$\rbrace\rbrace$ ;\\
~~\KwSty{static const double} D[6][3] = $\lbrace\lbrace$...$\rbrace\rbrace$ ;\\
~~\\
~~\KwSty{for} (\KwSty{int} i = 0; i < 6; ++i) $\lbrace$ \\
~~~~\KwSty{double} f0  = 0.0;\\
~~~~\KwSty{for} (\KwSty{int} r  = 0; r < 3; ++r) $\lbrace$ \\
~~~~~~f0 += (w[r] * C[i][r]);\\
~~~~$\rbrace$ \\
~~~~\KwSty{for} (\KwSty{int} j = 0; j < 3; ++j) $\lbrace$\\
~~~~~~\KwSty{for} (\KwSty{int} k = 0; k < 3; ++k) $\lbrace$\\
~~~~~~~~A[j][k] += (((((K[1]*B[i][k])+(K[3]*D[i][k])) * \\
~~~~~~~~~~~~~~~~~~~~~((K[1]*B[i][j])+(K[3]*D[i][j]))) + \\
~~~~~~~~~~~~~~~~~~~~(((K[0]*B[i][k])+(K[2]*D[i][k])) * \\
~~~~~~~~~~~~~~~~~~~~~((K[0]*B[i][j])+(K[2]*D[i][j]))))*det*W[i]*f0);\\
~~~~~~$\rbrace$\\
~~~~$\rbrace$\\
~~$\rbrace$\\
$\rbrace$
\caption{A possible implementation of Equation~\ref{eq:quadrature} assuming a 2D triangular mesh and polynomial order $q=2$ Lagrange basis functions.}
\label{code:weighted-laplace}
\end{algorithm}

More complex assembly expressions, due to the employment of particular differential operators in the original PDE, are obviously possible. Intuitively, as the complexity of the PDE grows, the implementation of local assembly becomes increasingly more complicated. This fact is actually the real motivation behind research in automated code generation techniques, such as those used by state-of-the-art frameworks like FEniCS and Firedrake. Automated code generation allows scientists to express the finite element specification using a domain-specific language resembling mathematical notation, and to obtain with minimum effort a semantically correct implementation of local assembly. The goal of this research is maximizing the efficiency, in terms of run-time performance, of generic local assembly kernels, on standard CPU architectures. 

\begin{algorithm}[t]
\scriptsize
\SetAlgorithmName{LISTING}{}

// This is a Firedrake construct (\textbf{not} an UFL's) to instantiate a 2D mesh.\\
mesh = UnitSquareMesh(size, size)\\
// FunctionSpace also belongs to the Firedrake language \\
V = FunctionSpace(mesh, "Lagrange", 2)\\
u = \KwSty{TrialFunction}(V)\\
v = \KwSty{TestFunction}(V)\\
weight = \KwSty{Function(V)}.\KwSty{assign}(value)\\
a = weight\KwSty{*}\KwSty{dot}(\KwSty{grad}(v), \KwSty{grad}(u))\KwSty{*dx}
\caption{UFL specification of the weighted Laplace equation for polynomial order $q=2$ Lagrange basis functions.}
\label{code:weighted-laplace-ufl}
\end{algorithm}

The domain-specific language used by Firedrake and FEniCS to express finite element problems is the Unified Form Language (UFL) \citep{ufl}. Listing~\ref{code:weighted-laplace-ufl} shows a possible UFL implementation for the weighted Laplace form. Note the resemblance of \emph{a = weight*...} with Equation~\ref{eq:stiffness}. A form compiler translates UFL code into the C code shown in Listing~\ref{code:weighted-laplace}. We will describe these aspects carefully in Section~\ref{sec:coffee-implementation}; for the moment, this level of detail suffices to open a discussion on the optimization of local assembly kernels arising from different partial differential equations.

% More Examples now...

\begin{figure}
\begin{alltt}
\scriptsize
\textbf{Input:} element matrix (2D array, initialized to 0), coordinates (array), 
       coefficients (array, e.g. velocity)
\textbf{Output:} element matrix (2D array)
- Compute Jacobian from coordinates
- Define basis functions
- Compute element matrix in an affine loop nest
\end{alltt}
\caption{Structure of a local assembly kernel}
\label{code:general-structure}
\end{figure}

%The code transformations described in this work are also
%generalizable to non-Firedrake assembly kernels, provided that the
%same numerical integration algorithm is employed.

The structure of a local assembly kernel can be generalized as in Figure~\ref{code:general-structure}. The inputs are a zero-initialized two dimensional array used to store the element matrix, the element's coordinates in the discretized domain, and coefficient fields, for instance indicating the values of velocity or pressure in the element. The output is the evaluated element matrix. The kernel body can be logically split into three parts:
\begin{enumerate}
  \item Calculation of the Jacobian matrix, its determinant and its
    inverse required for the aforementioned change of coordinates from
    the reference element to the one being computed.
  % FIXME: this rather hand-wavy and not technically correct
  \item Definition of basis functions used to interpolate fields at the
    quadrature points in the element. The choice of basis functions is
    expressed in UFL directly by users. In the generated code, they are
    represented as global read-only two dimensional arrays (i.e., using
    \texttt{static const} in C) of double precision floats.
  \item Evaluation of the element matrix in an affine loop nest, in which
    the integration is performed.
\end{enumerate}
Table~\ref{table:map-name-letters} shows the variable names we will use in the upcoming code snippets to refer to the various kernel objects.

\begin{table}
\scriptsize
\begin{center}
\begin{tabulary}{1.0\columnwidth}{C|C|C}
\hline
Object name & Type & Variable name(s) \\\hline
Determinant of the Jacobian matrix & double & det  \\
Inverse of the Jacobian matrix & double & K1, K2, ... \\
Coordinates & double** & coords\\
Fields (coefficients) & double** & w \\
Coefficients at quadrature points & double & f0, f1, ...\\
Numerical integration weights & double[] & W \\
Basis functions (and derivatives) & double[][] & A, B, C, ... \\
Element matrix & double[][] & M\\ \hline
\end{tabulary}
\end{center}
\caption{Type and variable names used in the various listings to identify local assembly objects.}
\label{table:map-name-letters}
\end{table}


\begin{algorithm}[t]
\scriptsize
\SetAlgorithmName{LISTING}{}

\KwSty{void} helmholtz(\KwSty{double} M[3][3], \KwSty{double} **coords) $\lbrace$\\
~~// K, det = Compute Jacobian (coords) \\
~~\\
~~\KwSty{static const double} W[3] = $\lbrace$...$\rbrace$\\
~~\KwSty{static const double} A[3][3] = $\lbrace\lbrace$...$\rbrace\rbrace$\\
~~\KwSty{static const double} B[3][3] = $\lbrace\lbrace$...$\rbrace\rbrace$\\
~~\\
~~\KwSty{for} (\KwSty{int} i = 0; i$<$3; i++) \\
~~~~\KwSty{for} (\KwSty{int} j = 0; j$<$3; j++) \\
~~~~~~\KwSty{for} (\KwSty{int} k = 0; k$<$3; k++) \\
~~~~~~~~M[j][k] += \underline{(}Y[i][k]*Y[i][j]+\\
~~~~~~~~~~~~~~~~~~~~~+((K1*A[i][k]+K3*B[i][k])*(K1*A[i][j]+K3*B[i][j]))+\\
~~~~~~~~~~~~~~~~~~~~~+((K0*A[i][k]+K2*B[i][k])*(K0*A[i][j]+K2*B[i][j]))\underline{)}*\\
~~~~~~~~~~~~~~~~~~~~~~*det*W[i];\\
$\rbrace$
\caption{Local assembly implementation for a Helmholtz problem on a 2D mesh using polynomial order $q=1$ Lagrange basis functions.}
\label{code:helmholtz}
\end{algorithm}

\begin{algorithm}[t]
\scriptsize
\SetAlgorithmName{LISTING}{}

\KwSty{void} burgers(\KwSty{double} A[12][12], \KwSty{double} **coords, \KwSty{double} **w) $\lbrace$\\
~~// K, det = Compute Jacobian (coords) \\
~~\\
~~\KwSty{static const double} W[5] = $\lbrace$...$\rbrace$\\
~~\KwSty{static const double} A[5][12] = $\lbrace\lbrace$...$\rbrace\rbrace$\\
~~\KwSty{static const double} B[5][12] = $\lbrace\lbrace$...$\rbrace\rbrace$\\
~~//11 other basis functions definitions.\\
~~...\\
~~\KwSty{for} (\KwSty{int} i = 0; i$<$5; i++) $\lbrace$\\
~~~~\KwSty{double} f0 = 0.0;\\
~~~~//10 other declarations (f1, f2,...)\\
~~~~...\\
~~~~\KwSty{for} (\KwSty{int} r = 0; r$<$12; r++) $\lbrace$\\
~~~~~~f0 += (w[r][0]*C[i][r]);\\
~~~~~~//10 analogous statements (f1, f2, ...)\\
~~~~~~...\\
~~~~$\rbrace$\\
~~~~\KwSty{for} (\KwSty{int} j = 0; j$<$12; j++) \\
~~~~~~\KwSty{for} (\KwSty{int} k = 0; k$<$12; k++) \\
~~~~~~~~A[j][k] += (..(K5*F9)+(K8*F10))*Y1[i][j])+\\
~~~~~~~~~~~~+(((K0*C[i][k])+(K3*D[i][k])+(K6*A[i][k]))*Y2[i][j]))*f11)+\\
~~~~~~~~~~~~+(((K2*E[i][k])+...+(K8*B[i][k]))*((K2*E[i][j])+...+(K8*B[i][j])))+\\
~~~~~~~~~~~~+ $<$roughly a hundred sum/muls go here$>$)..)*\\
~~~~~~~~~~~~*det*W[i];\\
~~$\rbrace$ \\
$\rbrace$
\caption{Local assembly implementation for a Burgers problem on a 3D mesh using polynomial order $q=1$ Lagrange basis functions.}
\label{code:burgers}
\end{algorithm}

The actual complexity of a local assembly kernel depends on the finite element problem being solved. In simpler cases, the loop nest is perfect, has short trip counts (in the range 3--15), and the computation reduces to a summation of a few products involving basis functions. An example is provided in Listing~\ref{code:helmholtz}, which shows the assembly kernel for a Helmholtz problem using Lagrange basis functions on 2D elements with polynomial order $q=1$. In other scenarios, for instance when solving the Burgers equation, the number of arrays involved in the computation of the element matrix can be much larger. The assembly code is given in Listing~\ref{code:burgers} and contains 14 unique arrays that are accessed, where the same array can be referenced multiple times within the same expression. This may also require the evaluation of constants in outer loops (called $F$ in the code) to act as scaling factors of arrays. Trip counts grow proportionally to the order of the method and arrays may be block-sparse.

In general, the variations in the structure of mathematical expressions and in loop trip counts (although typically limited to the order of tens of iterations) that different equations show, render the optimization process challenging, requiring distinct sets of transformations to bring performance closest to the machine peak. For example, the Burgers problem, given the large number of arrays accessed, suffers from high register pressure, whereas the Helmholtz equation does not. Moreover, arrays in Burgers are block-sparse due to the use of vector-valued basis functions (we will elaborate on this in the next sections). These few aspects (we could actually find more) already intuitively suggests that the two problems require a different treatment, based on an in-depth analysis of both data and iteration spaces. Furthermore, domain knowledge enables transformations that a general-purpose compiler could not apply, making the optimization space even larger. In this context, our goal is to understand the relationship between distinct code transformations, their impact on cross-loop arithmetic intensity, and to what extent their composability is effective in a wide class of real-world equations and architectures.

We also note that despite the infinite variety of assembly kernels that frameworks like FEniCS and Firedrake can generate, it is still possible to identify common domain-specific traits that are potentially exploitable for our optimization strategy. These include: 1) memory accesses along the three loop dimensions are always unit stride; 2) the \texttt{j} and \texttt{k} loops are interchangeable, whereas interchanges involving the $i$ loop require pre-computation of values (e.g. the $F$ values in Burgers) and introduction of temporary arrays (explained next); 3) depending on the problem being solved, the \texttt{j} and \texttt{k} loops could iterate along the same iteration space; 4) most of the sub-expressions on the right hand side of the element matrix computation depend on just two loops (either \texttt{i}-\texttt{j} or \texttt{i}-\texttt{k}). In the following sections we show how to exploit these observations to define a set of systematic, composable optimizations.

\subsection{Linear Solvers}
\label{sec:bkg:linearsolvers}
..

\subsection{Impact of Assembly on Execution Time}
\label{sec:bkg:impact}
...


\section{Abstractions in Computational Science}
\label{sec:bkg:abstractions}

\subsection{Domain Specific Languages}
...

\subsection{Multilayer Frameworks for the Finite Element Method}
\label{sec:bkg:firedrake}
Firedrake and FEniCS

\subsection{Abstractions for Mesh Iteration}
\label{sec:bkg:meshiteration}
\begin{description}
\item[Structured Meshes] ...
\item[Unstructured Meshes] OP2, PyOP2, Lizst
\end{description}

\subsubsection{OP2 and PyOP2}
\label{sec:bkg:op2}

\begin{algorithm}
\scriptsize\ttfamily
\SetAlgorithmName{LISTING}{}

void kernel1 (double * x, double * tmp1, double * tmp2) {
  *tmp1 += *x;
  *tmp2 += *x;
}
// loop over edges
op$\_$par$\_$loop (edges, kernel1,
  op$\_$arg$\_$dat (x, -1, OP$\_$ID, OP$\_$READ),
  op$\_$arg$\_$dat (temp, 0, edges2vertices, OP$\_$INC),
  op$\_$arg$\_$dat (temp, 1, edges2vertices, OP$\_$INC))

// loop over cells
op$\_$par$\_$loop (cells, kernel2,
  op$\_$arg$\_$dat (temp, 0, cells2vertices, OP$\_$INC),
  op$\_$arg$\_$dat (temp, 1, cells2vertices, OP$\_$INC),
  op$\_$arg$\_$dat (temp, 2, cells2vertices, OP$\_$INC),
  op$\_$arg$\_$dat (res, -1, OP$\_$ID, OP$\_$READ))

// loop over edges
op$\_$par$\_$loop (edges, kernel3,
  op$\_$arg$\_$dat (temp, 0, edges2vertices, OP$\_$INC),
  op$\_$arg$\_$dat (temp, 1, edges2vertices, OP$\_$INC))

\caption{...}
\label{code:op2program}
\end{algorithm}

%TODO : paste description from ESA report



\section{Compilers and Libraries for Code Optimization}
\label{sec:bkg:codeopt}


\subsection{Loop Optimization - Static Analysis}
\label{sec:bkg:poly}
Polyhedral compilers ...
scheduling functions
Mention their unsuitability for tiling unstructured meshes... (use email I sent listing all issues...)




\subsection{Loop Optimization - Dynamic Analysis}
\label{sec:bkg:ie}
\begin{itemize}
\item Inspector/Executor schemes
\item Space filling curves for unstructured meshes
\item sparse tiling = loop fusion + loop tiling
\end{itemize}


\subsection{On the Dichotomy between Loop Tiling and Loop Fusion}
\label{sec:bkg:tiling}

\begin{itemize}
\item Tiling is for a nested loop ! 
\item tiles are atomic blabla
\item split vs overlapped
\end{itemize}


\subsubsection{Split tiling}


\subsubsection{Overlapped tiling}
 %These techniques work by dividing a loop nest's iteration space into a number of overlapping regions called tiles. These tiles are shaped such that each of them can execute in parallel without requiring communication.  
 
 \subsubsection{Fusion vs Tiling}
 \label{sec:bkg:sparsetiling}
 so time tiling is just fusion across the time stepping loop, then tiling
% IDEA: time skewing is just fusion. If it's a nested loop nest, you do tiling; but across loops at the same level of a block, you always do fusion. This way, SFCs and tiling are instances of scheduling functions for a nested loop; what is called "time tiling" or "time skewing" is just a shorthand for "fuse the loops within the time stepping loop".

%Sparse tiling techniques were developed to group iterations of irregular applications
%into atomic tiles at runtime with an 
%inspector~\citep{dimeEtna00,StroutIJHPCA,StroutPLDI03,commAvoidingSparse2009}.
%In general, the inspector iterates over index arrays that do not change during
%the main computation to determine data reorderings or new schedules, like
%sparse tiling schedules.
%The resulting tiles have either an implicit or explicit partial ordering (i.e., a task graph)
%that exposes asynchronous parallelism~\citep{Adams99c,dimeEtna00,StroutLCPC2002}.
%These benchmark-specific, sparse tiling executors
%exhibited performance improvements for sparse stencil computations, 
%%FIXME: dime?Jacobi, % FIXME:~\cite{dime?}, 
%Gauss-Seidel~\citep{Adams99c,StroutIJHPCA}, moldyn~\cite{StroutPLDI03},
%and sparse matrix powers kernel~\cite{commAvoidingSparse2009}.

%\subsection{Space Filling Curves}
%...

\subsection{Domain Specific Optimization}
...

\subsubsection{Tensor Contraction Engine}
...

\subsubsection{LGen}
Why potentially useful ...

\subsubsection{Halide}
...

\subsubsection{Spiral}
...

\section{State-of-the-art Hardware Architectures}
\label{sec:bkg:arch}
...

\subsection{SIMD Vectorization}
...

\subsection{Terminology}
\label{sec:bkg:terminology}

\begin{description}
\item[Memory pressure, Register pressure]
\item[Arithmetic intensity]
\item[Operational intensity] Maybe add a subsection of the roofline model ?
\item[Flops]
\item[Access function (for array)]
\item[General-purpose compiler]
\item[Communication vs computation]
\item[CPU- vs Memory-bound]
\item[Local vs Global reduction]
\end{description}
