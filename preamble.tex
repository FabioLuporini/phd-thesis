% vi:tw=78:fo=aw2tq

%%%%%%%%%%%%%%%%%%%%%%%%%%%%%%%%%%%%%%%%%%%%%%%%%%%%%%%%%%%%%%%%%%%%
%
% IMPERIAL COLLEGE LONDON DISSERTATION TEMPLATE
%
%%%%%%%%%%%%%%%%%%%%%%%%%%%%%%%%%%%%%%%%%%%%%%%%%%%%%%%%%%%%%%%%%%%%
%
% This file is `icldiss.tex'
%
% This document fulfills the layout requirements for Dissertations
% of the University of London and of Imperial College London.
% To do so it uses the documentclass `icldt' which is provided free
% of charge under the MIT license. The relevant College regulations,
% and the license are included in the `icldt' manual.
%
% If you print your dissertation for yourself or as a present for
% family, friends or colleagues you probably should use a different
% layout which does not fulfill the College requirements but which
% can look much better.
%
% For further information and for professional layouting and
% printing services please visit www.PrettyPrinting.net
%
% Copyright (c) 2008, Daniel Wagner, www.PrettyPrinting.net
%
%%%%%%%%%%%%%%%%%%%%%%%%%%%%%%%%%%%%%%%%%%%%%%%%%%%%%%%%%%%%%%%%%%%%

\documentclass[titlepagenumber=off,twoside,openright,declaration=on]{icldt}
\usepackage[utf8]{inputenc}
\usepackage[sc]{mathpazo}
\usepackage[scaled=0.90]{helvet}
\usepackage{inconsolata}
\usepackage{courier}

\usepackage{amsfonts}
\usepackage[cmex10]{amsmath}
\usepackage{amssymb}

\usepackage{fncychap}
\usepackage{fancyhdr}

\usepackage{graphicx}
\usepackage{subcaption}
\usepackage[font=footnotesize,labelfont=bf]{caption}
%\usepackage{subfigure}

\usepackage{url}
\usepackage{xfrac}

\usepackage{natbib}

\usepackage[ruled, longend, linesnumbered]{algorithm2e}
\SetKwInput{kwInput}{Input}
\SetKwInput{kwOutput}{Output}
\SetKw{Not}{not}
\SetKw{True}{true}
\SetKw{False}{false}
\SetKw{Skip}{skip}
\SetKwRepeat{Do}{do}{while}
\SetKwComment{Comment}{// }{}
\let\oldnl\nl% Store \nl in \oldnl
\newcommand{\nonl}{\renewcommand{\nl}{\let\nl\oldnl}}% Remove line number for one line

\renewcommand{\algorithmcfname}{ALGORITHM}
\SetAlFnt{\small}
\SetAlCapFnt{\small}
\SetAlCapNameFnt{\small}
\SetAlCapHSkip{0pt}
\IncMargin{-\parindent}

\usepackage{alltt}
\renewcommand{\ttdefault}{txtt}

\usepackage{tabulary}
\newcommand{\specialcell}[2][c]{%
  \begin{tabular}[#1]{@{}c@{}}#2\end{tabular}}
\usepackage{multicol}

\usepackage{array}
\newcolumntype{P}[1]{>{\centering\arraybackslash}p{#1}}

\usepackage{lineno}

% Notation macros
\newcommand{\dx}{\, \mathrm{d}x}
\newcommand{\dX}{\, \mathrm{d}X}
\newcommand{\ds}{\, \mathrm{d}s}
\newcommand{\dS}{\, \mathrm{d}S}
\newcommand{\dt}{\, \mathrm{d}t}
\newcommand{\R}{\mathbb{R}}
\newcommand{\Hdiv}{H(\mathrm{div})}
\newcommand{\Hcurl}{H(\mathrm{curl})}
\newcommand{\nedelec}{N\'ed\'elec}
\newcommand{\babuska}{Babu\v{s}ka}
\newcommand{\tab}{\hspace*{2em}}
\newcommand{\Rset}{\ensuremath{\mathbb{R}}}
\newcommand{\inner}[2]{\langle #1, #2 \rangle}
\newcommand{\deltat}{\Delta t}
\newcommand{\abs}[1]{\mathopen| #1 \mathclose|}   % use instead of $|x|$
\newcommand{\pp}[2]{\frac{\partial #1}{\partial #2}}


%%%%%%%%%%%%%%%%%%%%%%%%%%%%%%%%%%%%%%%%%%%%%%%%%%%%%%%%%%%%%%%%%%%%%%%%%%%%%%%%%%%%%%%%%%%%
%%%%%%%%%%%%%%%%%%%% REQUIRED FOR GENERALIZED SPARSE TILING SECTION %%%%%%%%%%%%%%%%%%%%%%%%
%%%%%%%%%%%%%%%%%%%%%%%%%%%%%%%%%%%%%%%%%%%%%%%%%%%%%%%%%%%%%%%%%%%%%%%%%%%%%%%%%%%%%%%%%%%%

\newcommand{\oploop}{op\_par\_loop }
\newcommand{\oploopn}{op\_par\_loop}
\newcommand{\oploops}{op\_par\_loops }
\newcommand{\oploopsn}{op\_par\_loops}

\newcommand{\opset}{op\_set }
\newcommand{\opsetn}{op\_set}
\newcommand{\opsets}{op\_sets }
\newcommand{\opsetsn}{op\_sets}

\newcommand{\opmap}{op\_map }
\newcommand{\opmapn}{op\_map}
\newcommand{\opmaps}{op\_maps }
\newcommand{\opmapsn}{op\_maps}

\newcommand{\opdat}{op\_dat }
\newcommand{\opdatn}{op\_dat}
\newcommand{\opdats}{op\_dats }
\newcommand{\opdatsn}{op\_dats}

\newcommand{\opdset}{op\_decl\_set }
\newcommand{\opdsetn}{op\_decl\_set}
\newcommand{\opdmap}{op\_decl\_map }
\newcommand{\opdmapn}{op\_decl\_map}
\newcommand{\opddat}{op\_decl\_dat }
\newcommand{\opddatn}{op\_decl\_dat}
\newcommand{\oparg}{op\_arg }
\newcommand{\opargn}{op\_arg}
\newcommand{\opargs}{op\_args }
\newcommand{\opargsn}{op\_args}
\newcommand{\opargd}{op\_arg_dat }
\newcommand{\opargdn}{op\_arg_dat}

%\usepackage{cite}

\DeclareMathOperator*{\Max}{\textsl{MAX}}

\newcommand{\code}[1]{\lstset{basicstyle=\small\tt}\lstinline£#1£\lstset
{basicstyle=\footnotesize\tt}}


\makeatletter
\newenvironment{CenteredBox}{% 
\begin{Sbox}}{% Save the content in a box
\end{Sbox}\centerline{\parbox{\wd\@Sbox}{\TheSbox}}}% And output it centered
\makeatother


\usepackage{setspace}
\usepackage{fancybox}
% \usepackage{enumitem}


\usepackage{color}
\usepackage{enumerate}

\usepackage{listings}
\lstset{
	language=C++,
	stepnumber=1,
	basicstyle=\tiny\ttfamily,
	numbersep=5pt,
	backgroundcolor=\color{white},
	showspaces=false,
%	showstringspaces=false,
	showtabs=false,
	tabsize=2,
	breaklines=true,
	breakatwhitespace=false,
	mathescape,
	morekeywords={each,min,not},
	escapeinside={\%*}{*)}
}

\let\proof\relax
\let\endproof\relax
\usepackage{amsthm}


% Theorems, definitions, lemmas, etc
\newtheorem{Def}{Definition}
\newtheorem{Prop}{Proposition}
\newtheorem{Algo}{Algorithm}
\newtheorem{Const}{Constraint}
\newtheorem{Strategy}{Strategy}


\DeclareMathOperator*{\argmin}{arg\,min}

% The following configs are thanks to Florian Rathgeber
\usepackage[usenames,dvipsnames,svgnames]{xcolor}
\usepackage[pdftex,
            urlcolor=DarkBlue,       % \href{...}{...} external (URL)
            filecolor=DarkGreen,     % \href{...} local file
            linkcolor=DarkRed,       % \ref{...} and \pageref{...}
            citecolor=DarkGreen,
            %allcolors=black,
            colorlinks=true,
            hypertexnames=false,
            bookmarks=true,
            bookmarksnumbered=true,
            bookmarksopen=false]{hyperref}

%\usepackage{enumitem}
%\setlist[itemize]{itemsep=0pt,topsep=7pt,leftmargin=12pt}
%\setlist[enumerate]{itemsep=0pt,topsep=0pt}
%\setlist[description]{itemsep=0pt,topsep=0pt,leftmargin=0pt}

% To overwrite Latex's stupid rules for floats
\renewcommand{\topfraction}{.85}
\renewcommand{\bottomfraction}{.7}
\renewcommand{\textfraction}{.15}
\renewcommand{\floatpagefraction}{.66}
\renewcommand{\dbltopfraction}{.66}
\renewcommand{\dblfloatpagefraction}{.66}
\setcounter{topnumber}{9}
\setcounter{bottomnumber}{9}
\setcounter{totalnumber}{20}
\setcounter{dbltopnumber}{9}