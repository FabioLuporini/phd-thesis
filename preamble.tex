% vi:tw=78:fo=aw2tq

%%%%%%%%%%%%%%%%%%%%%%%%%%%%%%%%%%%%%%%%%%%%%%%%%%%%%%%%%%%%%%%%%%%%
%
% IMPERIAL COLLEGE LONDON DISSERTATION TEMPLATE
%
%%%%%%%%%%%%%%%%%%%%%%%%%%%%%%%%%%%%%%%%%%%%%%%%%%%%%%%%%%%%%%%%%%%%
%
% This file is `icldiss.tex'
%
% This document fulfills the layout requirements for Dissertations
% of the University of London and of Imperial College London.
% To do so it uses the documentclass `icldt' which is provided free
% of charge under the MIT license. The relevant College regulations,
% and the license are included in the `icldt' manual.
%
% If you print your dissertation for yourself or as a present for
% family, friends or colleagues you probably should use a different
% layout which does not fulfill the College requirements but which
% can look much better.
%
% For further information and for professional layouting and
% printing services please visit www.PrettyPrinting.net
%
% Copyright (c) 2008, Daniel Wagner, www.PrettyPrinting.net
%
%%%%%%%%%%%%%%%%%%%%%%%%%%%%%%%%%%%%%%%%%%%%%%%%%%%%%%%%%%%%%%%%%%%%

\documentclass[titlepagenumber=off,twoside,openright,declaration=on]{icldt}
\usepackage[utf8]{inputenc}

\usepackage{amsfonts}
\usepackage{amsmath}
\usepackage{amssymb}

\usepackage{fncychap}
\usepackage{fancyhdr}
\usepackage{graphicx}
\usepackage{subfigure}
\usepackage{xfrac}

\usepackage{natbib}

\usepackage{listings}
\lstset{language=C, breaklines=true}

\usepackage[ruled]{algorithm2e}
\renewcommand{\algorithmcfname}{ALGORITHM}
\SetAlFnt{\small}
\SetAlCapFnt{\small}
\SetAlCapNameFnt{\small}
\SetAlCapHSkip{0pt}
\IncMargin{-\parindent}

\usepackage{alltt}
\renewcommand{\ttdefault}{txtt}

\usepackage{tabulary}
\newcommand{\specialcell}[2][c]{%
  \begin{tabular}[#1]{@{}c@{}}#2\end{tabular}}
\usepackage{multicol}

\usepackage{lineno}

% Notation macros
\newcommand{\dx}{\, \mathrm{d}x}
\newcommand{\dX}{\, \mathrm{d}X}
\newcommand{\ds}{\, \mathrm{d}s}
\newcommand{\dS}{\, \mathrm{d}S}
\newcommand{\dt}{\, \mathrm{d}t}
\newcommand{\R}{\mathbb{R}}
\newcommand{\Hdiv}{H(\mathrm{div})}
\newcommand{\Hcurl}{H(\mathrm{curl})}
\newcommand{\nedelec}{N\'ed\'elec}
\newcommand{\babuska}{Babu\v{s}ka}
\newcommand{\tab}{\hspace*{2em}}
\newcommand{\Rset}{\ensuremath{\mathbb{R}}}
\newcommand{\inner}[2]{\langle #1, #2 \rangle}
\newcommand{\deltat}{\Delta t}
\newcommand{\abs}[1]{\mathopen| #1 \mathclose|}   % use instead of $|x|$
\newcommand{\pp}[2]{\frac{\partial #1}{\partial #2}}
