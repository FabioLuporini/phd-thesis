\chapter{On Optimality of Finite Element Integration}
\label{ch:optimality}

\section{Introduction}
...

\section{Rewrite Operators}
...
%\subsection{Generalized Loop-invariant Code Motion}
%\label{sec:coffee-licm}
%
%\begin{figure}
%\centering
%\footnotesize
%\lstinputlisting{coffee/listings/simple.code}
%\caption{Excerpt of the non-optimized local assembly code from the Burgers problem shown in Listing~\ref{code:burgers}.}
%\label{code:original-code}
%\end{figure}
%
%Consider the local element matrix computation in Figure~\ref{code:original-code}, which is an excerpt from the Burgers problem shown in Listing~\ref{code:burgers}. The assembly expression, produced by the FEniCS and Firedrake's form compiler, has been deliberately simplified, and code details have been omitted for brevity and readability. In practice, as already emphasized, assembly expressions can be much more complex depending on the differential operators employed in the variational form; however, this example is representative enough for highlighting patterns that are common in a large class of equations. 
%
%A first glimpse of the code suggests that the \texttt{a*f0*A[i][j]+b*f1*B[i][j]} sub-expression is invariant with respect to the innermost (trial functions) loop \texttt{k}, so it can be hoisted at the level of the outer loop \texttt{j} to avoid redundant computation. This is indeed a standard compiler transformation, supported by any available compilers, so, in principle, there should be no need to transform the source code explicitly. With a closer look we notice that the sub-expression \texttt{d*D[i][k]+e*E[i][k]} is also invariant, although, this time, with respect to the outer (test functions) loop \texttt{j}. Available compilers (e.g. \emph{GNU's} and \emph{Intel's}) limit the search for code motion opportunities to the innermost loop of a given nest. Moreover, the hoisted code is scalar and therefore not subjected to SIMD auto-vectorization. In other words, these general-purpose compilers lack performance models to determine (i) the optimal place where to hoist an expression and (ii) the potential gain and overhead (due to the need for extra temporary memory) of vectorization. These are notable limitations for local assembly kernels. 
%
%\begin{figure}
%\centering
%\footnotesize
%\lstinputlisting{coffee/listings/invariant.code}
%\caption{Local assembly code for the Burgers example after application of generalized loop-invariant code motion.}
%\label{code:invariant-code}
%\end{figure}
%
%We work around these limitations with source-level loop-invariant code motion. In particular, we pre-compute all values that an invariant sub-expression assumes along its fastest varying dimension. This is implemented by introducing a temporary array per invariant sub-expression and by adding a new loop to the nest. At the price of extra memory for storing temporaries, the gain is that lifted terms can be auto-vectorized as part of an inner loop. Given the short trip counts of our loops, it is important to achieve auto-vectorization of hoisted terms in order to minimize the percentage of scalar instructions, which could otherwise be significant. It is also worth noting that, in some problems, for instance Helmholtz in Listing~\ref{code:helmholtz}, invariant sub-expressions along \texttt{j} are identical to those along \texttt{k}, and both loops iterate over the same iteration space, as anticipated in Section~\ref{sec:coffe-motivations}. In these cases, we safely avoid redundant pre-computation. The resulting code for the running Burgers example is shown in Figure~\ref{code:invariant-code}.
%
%In the following, we refer to this series of transformations as \textit{generalized loop-invariant code motion}. We will show that this optimization is crucial when optimizing non-trivial assembly expressions, allowing to achieve performance improvements over the original code larger than 3$\times$.
%
%
%\subsection{Terms Factorization}
%\label{sec:coffee-factorize}
%
%\begin{figure}
%\centering
%\footnotesize
%\lstinputlisting{coffee/listings/factorized.code}
%\caption{Terms factorization exposes additional code hoisting opportunities, as it can be evinced by comparison of the code in this Figure with that in Figure~\ref{code:invariant-code}.}
%\label{code:factorized-code}
%\end{figure}
%
%After generalized loop-invariant code motion has been applied, some assembly expressions can still  ``hide'' opportunities for code hoisting. By examining again the code in Figure~\ref{code:invariant-code}, we notice that the basis function array \texttt{A} iterating along the \texttt{[i,j]} loops appears twice in the expression. By expanding the products in which \texttt{A} is accessed and by applying sum commutativity, terms can be factorized. This has two effects: firstly, it reduces the number of arithmetic operations performed; secondly, and most importantly, it exposes a new sub-expression \texttt{A[i][k]/c+T2[k]*f} invariant with respect to loop \texttt{j}. Consequently, hoisting can be performed, resulting in the code in Figure~\ref{code:factorized-code}. In general, exposing factorization opportunities requires traversing the whole expression tree, and then expanding and moving terms. It also needs heuristics to select a factorization strategy: there may be different opportunities of reorganizing sub-expressions, and, in our case, the best is the one that maximizes the invariant code eventually disclosed. We will discuss this aspect formally in Section~\ref{sec:coffee-rewrite-rules}.
%
%
%\subsection{Expanding Sub-expressions}
%\label{sec:coffee-expansion}
%
%Expression rewriting also aims at minimizing register pressure in the assembly loop nest. Once the code has been optimized for arithmetic intensity, it is important to think about how the transformations impacted register allocation. Assume the local assembly kernel is executed on a state-of-the-art CPU architecture having 16 logical registers, e.g. an Intel Haswell. Each value appearing in the expression is loaded and kept in a register as long as possible. In Figure~\ref{code:scalarexp-code}, for instance, the scalar value \texttt{g} is loaded once, whereas the term \texttt{det*W[i]} is precomputed and loaded in a register at every \texttt{i} iteration. This implies that at every iteration of the \texttt{[j,k]} loop nest, 12$\%$ of the available registers are spent just to store values independent of test and trial functions loops. In more complicated expressions, the percentage of registers destined to store such constant terms can be even higher. Registers are, however, a precious resource, especially when evaluating compute-intensive expressions. The smaller is the number of available free registers, the worse is the instruction-level parallelism achieved: for example, a shortage of registers can increase the pressure on the L1 cache (i.e. it can worsen data locality), or it may prevent the effective application of standard transformations, e.g. loop unrolling. We aim at relieving this problem by suitably expanding terms and introducing, where necessary, additional temporary values. We illustrate this in the following example.
%
%%An analogous analysis applies to processors with larger numbers of registers, since using loop unroll or loop unroll-and-jam to expose more instruction-level parallelism would increase the requirements on registers.
%
%\begin{figure}
%\footnotesize
%\subfigure[Excerpt of the non-optimized local assembly code from the Burgers problem shown in Listing~\ref{code:burgers}, slightly modified for illustration purposes.]{\label{code:toexpand-code}\lstinputlisting{coffee/listings/toexpand.code}}
%~\\
%~\\
%~\\
%~\\
%\subfigure[Local assembly code after expansion of term \texttt{det*W[i]}]{\label{code:expanded-1-code}\lstinputlisting{coffee/listings/expanded-1.code}}
%~\\
%~\\
%~\\
%~\\
%\subfigure[Local assembly code after expansion of symbol \texttt{g}. Note the need to introduce a new temporary array.]{\label{code:expanded-2-code}\lstinputlisting{coffee/listings/expanded-2.code}}
%\caption{Expansion of terms to improve register pressure in a local assembly kernel}\label{code:expanded-code}
%\end{figure}
%
%Consider a variant of the Burgers local assembly kernel, shown in Figure~\ref{code:toexpand-code}. This is again a representative, simplified example. We can easily distribute \texttt{det*W[i]} over the three operands on the left-hand side of the multiplication, and then absorb it in the pre-computation of the invariant sub-expression stored in \texttt{T1}, resulting in code as in Figure~\ref{code:expanded-1-code}. Freeing the register destined to the constant \texttt{g} is less straightforward: we cannot absorb it in \texttt{T1} as we did with \texttt{det*W[i]} since \texttt{T1} is also accessed in the \texttt{T1[j]*A[i][k]} sub-expression. The solution is to add another temporary as in Figure~\ref{code:expanded-2-code}. Generalizing, this is a problem of data dependencies: to solve it, we use a dependency graph in which we add a direct edge from identifier \texttt{A} to identifier \texttt{B} to denote that the evaluation of \texttt{B} depends on \texttt{A}. The dependency graph is initially empty, and is updated every time a new temporary is created by either loop-invariant code motion or expansion of terms. The dependency graph is then queried to understand when expansion can be performed without resorting to new temporary values. This aspect is formalized in the next section.


% ***********************************************************************************************************************************************

%\section{Avoiding Iteration over Zero-valued Blocks by Symbolic Execution}
%\label{sec:coffee-avoidzeros}
...
%Skipping arithmetic operations over blocks of zero-valued entries in basis functions arrays is the second goal of expression rewriting. Zero-valued columns arise, for example, when taking derivatives on a reference element and when employing vector-valued elements. In~\cite{quadrature-olegaard}, a technique to avoid operations on zero-valued columns based on the use of indirection arrays (e.g. \texttt{A[B[i]]}, where \texttt{A} is a tabulated basis function and \texttt{B} a map from loop iterations to non-zero columns in \texttt{A}) was described and implemented in FEniCS. The approach proposed in this section will be evaluated and compared to this pioneering work. Essentially, our strategy avoids indirection arrays in the generated code, which otherwise would break the optimizations applicable by code specialization, including SIMD vectorization. 
%
%\begin{figure}[t]
%\tiny
%\centerline{
%\subfigure[]{\label{fig:withzeros-code}\lstinputlisting{coffee/listings/withzeros.code}}
%~~~~~~~~~
%\subfigure[]{\label{fig:withzeros-skipped-code}\lstinputlisting{coffee/listings/skipzeros.code}}
%}
%\caption{On the left, excerpt of the non-optimized local assembly code from the Burgers problem shown in Listing~\ref{code:burgers}. On the right, the code after the assembly expression's iteration space has been restructured, based on the propagation of zero-valued columns as determined through symbolic execution. Note the code annotation over the definition of the tabulated vector-valued basis function \texttt{D}, which is provided as input to identify zero-valued columns.}\label{fig:skip-code}
%\end{figure}
%
%In Figure~\ref{fig:withzeros-code}, an enriched version of the Burgers excerpt in Figure~\ref{code:original-code} is illustrated. The code is instantiated for the specific case of polynomial order $q=1$ Lagrange basis functions on a 2D mesh. The array \texttt{D} represents a derivative of a basis function tabulated at the various quadrature points. There are four zero-valued columns. Any multiplications or additions along these columns could (should) be skipped to avoid irrelevant floating point operations. The solution adopted in~\cite{quadrature-olegaard} is not to generate the zero-valued columns (i.e. to generate a dense 6$\times$2 array), to reduce the size of the iteration space over test and trial functions (from 6 to 2), and to use an indirection array (e.g. $ind = \lbrace 3, 5\rbrace$) to update the right entries in the element tensor $A$. This prevents, among the various optimizations, effective SIMD vectorization, because memory loads and store would eventually reference non-contiguous locations. 
%%on the other hand, reshrinki tutto su un array solo...
%
%Our new strategy exploits domain knowledge and makes use of symbolic execution. We discern the origin of zero-valued columns: for example, those due to taking derivatives on the reference element from those inherent to using vector-valued elements. In the running Burgers example, the use of vector-valued basis functions required the introduction of a zero-valued block (columns 0, 1, 2 in the array \texttt{D}) to correctly evaluate the local element matrix while iterating along the space of test and trial functions. The two key observations are that: (i) the number of zero-valued columns caused by using vector function spaces is, often, much larger then that due to derivatives, and (ii) such columns are contiguous in memory. Based on this observation, we aim to avoid iteration only along the block of zero-valued columns induced by vector-valued elements. 
%
%%Our example is then transformed as in Figure~\ref{fig:withzeros-skipped-code}: loop bounds are adjusted and suitable offsets are introduce to access the element matrix and basis function arrays. In general, the element matrix evaluation may also have to be split over multiple iteration spaces (test and trial functions loops), each iteration space characterized by its own loop bounds; this has the side effect of both increasing loop overhead and decreasing data locality.
%
%The goal is achieved by means of symbolic execution. The algorithm expects some indication about the location of the zero-valued columns induced by vector-valued function spaces, for each tabulated basis function, as shown in Figure~\ref{fig:withzeros-code}. Then, after expression rewriting took place, each statement is executed symbolically. For example, consider the assignment \texttt{T2[r] = d*D[i][k]+e*E[i][k]} in Figure~\ref{fig:withzeros-code}. Array \texttt{D} has non-zero-valued columns in the range $NZ_D=[3,5]$, while array \texttt{E} has non-zero-valued columns in the range $NZ_E=[0,2]$, although not displayed. Multiplications by scalar quantities (e.g. \texttt{d*D[i][k]}) do not affect the propagation of non-zero-valued columns. On the other hand, summing two operands such as \texttt{d*D[i][k]} and \texttt{e*E[i][k]} requires tracking the fact that the target identifier \texttt{T2} will have non-zero-valued columns in the range $NZ_E \| NZ_D=[0,5]$. Eventually, exploiting the $NZ$ information computed and associated with each identifier, we split the original assembly expression into multiple sets of sub-expressions, each set characterized by the same range of non-zero-valued columns. In our example, assuming that $NZ_{T1}=[3,5]$ and $NZ_A=[3,5]$, there are two of such sets, which leads to the generation of two distinct iteration spaces (one for each set), as in Figure~\ref{fig:withzeros-skipped-code}.

%Code specialization's goal is architecture-specific optimization for instruction-level parallelism and register locality. A number of transformations are provided at this stage, including those enabling effective SIMD vectorization. 