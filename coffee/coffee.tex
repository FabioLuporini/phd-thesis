\chapter{COFFEE: an Optimizing Compiler for Finite Element Local Assembly}
\section{Introduction and Motivations}
In many fields such as computational fluid dynamics, computational electromagnetics and structural mechanics, phenomena are modelled by partial differential equations (PDEs). Numerical techniques, like the finite volume method and the finite element method, are widely employed to approximate solutions of these PDEs. Unstructured meshes are often used to discretize the computational domain, since they allow an accurate representation of complex geometries. The solution is sought by applying suitable numerical operations, or
kernels, to the entities of a mesh, such as edges, vertices, or cells. On standard clusters of multicores, typically, a kernel is executed sequentially by a thread, while parallelism is achieved by partitioning the mesh and assigning each partition to a different node or thread. Such an execution model, with minor variations, is adopted, for example, in \cite{pyop2isc}, \cite{Fenics}, \cite{fluidity_manual_v4}, \cite{lizst}.

The time required to apply the numerical kernels is a major issue, since the equation domain needs to be discretized into an extremely large number of cells to obtain a satisfactory approximation of the PDE, possibly of the order of trillions, as in \cite{Rossinelli2013}. For example, it has been well established that mesh resolution is critical in the accuracy of numerical weather forecasts. However, operational forecast centers have a strict time limit in which to produce a forecast - 60 minutes in the case of the UK Met Office. Producing efficient kernels has a direct scientific payoff in higher resolution, and therefore more accurate, forecasts. Computational cost is a dominant problem in computational science simulations, especially for those based on finite elements, which are the subject of this work. In this chapter, we address, in particular, the well-known problem of optimizing the local assembly phase of the finite element method, which can be responsible for a significant fraction of the overall computation run-time, often in the range 30-60$\%$ \cite{Francis,quadrature1,petsc-integration-gpu,Kirby-FEM-opt}. 

During the assembly phase, the solution of the PDE is approximated by executing a problem-specific kernel over all cells, or elements, in the discretized domain. We restrict our focus to relatively low order finite element methods, in which an assembly kernel's working set is usually small enough to fit the L1 cache. Low order methods are by no means exotic: they are employed in a wide variety of fields, including climate and ocean modeling, computational fluid dynamics, and structural mechanics. High order methods such as the spectral element method \cite{SPENCER} commonly require different algorithms to solve PDEs, so they are excluded from our study.

An assembly kernel is characterized by the presence of an affine, often non-perfect loop nest, in which individual loops are rather small: their trip count rarely exceeds 30, and may be as low as 3 for low order methods. In the innermost loop, a problem-specific, compute intensive expression evaluates a two dimensional array, representing the result of local assembly in an element of the discretized domain. With such a kernel structure, we focus on aspects like minimization of floating-point operations, register allocation and instruction-level parallelism, especially in the form of SIMD vectorization.

Achieving high-performance is non-trivial. The complexity of the mathematical expressions, often characterized by a large number of operations on constants and small matrices, makes it hard to determine a single or specific sequence of transformations that is successfully applicable to all problems. Loop trip counts are typically small and can vary significantly, which further exacerbates the issue. We will show that traditional vendor compilers, such as \emph{GNU's} and \emph{Intel's}, fail at exploiting the structure inherent such assembly expressions. Polyhedral-model-based source-to-source compilers, for instance~\cite{PLUTO}, can apply aggressive loop optimizations, such as tiling, but these are not particularly helpful in our context, as explained next. 

We focus on optimizing the performance of local assembly operations produced by automated code generation. This technique has been proved successful in the context of the FEniCS~\cite{Fenics} and Firedrake~\cite{firedrake-code} projects, become incredibly popular over the last years. In these frameworks, a mathematical model is expressed at high-level by means of a domain-specific language and a domain-specific compiler is used to produce a representation of local assembly operations (e.g. C code). Our aim is to obtain close-to-peak performance in all of the local assembly operations that such frameworks can produce. Since the domain-specific language exposed to the users provide as constructs generic differential operators, an incredibly vast set of PDEs, possibly arising in completely different domains, can be expressed and solved. A compiler-based approach is, therefore, the only reasonable option to the problem of optimizing local assembly operations. 

Several studies have already tackled local assembly optimization in the context of automated code generation. In~\cite{quadrature1}, it is shown how automated code generation can be leveraged to introduce domain-specific optimizations, which a user cannot be expected to write ``by hand''. \cite{Kirby-FEM-opt} and~\cite{Francis} have studied, instead, different optimization techniques based on a mathematical reformulation of the local assembly operations. The same problem has been addressed recently also for GPU architectures, for instance in~\cite{petsc-integration-gpu},~\cite{Klockner}, and~\cite{Bana}. With our study, we make clear step forward by showing that different PDEs, on different platforms, require distinct sets of transformations if close-to-peak performance must be reached, and that low-level, domain-aware code transformations are essential to maximize instruction-level parallelism and register locality. As discussed in the following sections, our optimization strategy is quite different from those in previous work, although we reuse and leverage some of the ideas formulated in the available literature. 

We present a novel structured approach to the optimization of automatically-generated local assembly kernels. We argue that for complex, realistic PDEs, peak performance can be achieved only by passing through a two-step optimization procedure: 1) expression rewriting, to minimize floating point operations, 2) and code specialization, to ensure effective register utilization and instruction-level parallelism, especially SIMD vectorization. 

Expression rewriting consists of a framework capable of minimizing arithmetic intensity and optimize for register pressure. Our contributions is twofold
\begin{itemize}
\item \emph{Rewrite rules for assembly expressions}. The goal is to the reduce the computational intensity of local assembly kernels by rescheduling arithmetic operations based on a set of rewrite rules. These aggressively exploit associativity, distributivity, and commutativity of operators to expose loop-invariant sub-expressions and SIMD vectorization opportunities to the code specialization stage. While rewriting an assembly expression, domain knowledge is used in several ways, for example to avoid redundant computation.
\item \emph{An algorithm to deschedule useless operations}. Relying on symbolic execution, this algorithm restructures the code so as to skip useless arithmetic operations, for example multiplication by scalar quantities which are statically known to be zero. One problem is to transform the code while preserving code vectorizability, which is solved by resorting to domain-knowledge. 
\end{itemize}

Code specialization's goal is to apply transformations to maximize the exploitation of the underlying platform's resources, e.g. SIMD lanes. We provide a number of contributions
\begin{itemize}
\item \emph{Padding and data alignment}. The small size of the loop nest (integration, test, and trial functions loops) require all of the involved arrays to be padded to a multiple of the vector register length so as to maximize the effectiveness of SIMD code. Data alignment can be enforced as a consequence of padding. 
\item \emph{Vector-register Tiling}. Blocking at the level of vector registers, which we perform exploiting the specific memory access pattern of the assembly expressions (i.e. a domain-aware transformation), improves data locality beyond traditional unroll-and-jam optimizations. This is especially true for relatively high polynomial order (i.e. greater than 2) or when pre-multiplying functions are present.
\item \emph{Expression Splitting}. In certain assembly expressions the register pressure is significantly high: when the number of basis functions arrays (or, equivalently, temporaries introduced by loop-invariant code motion) and constants is large, spilling to L1 cache is a consequence for architectures with a relatively low number of logical registers (e.g. 16/32). We exploit sum's associativity to ``split'' the assembly expression into multiple sub-expressions, which are computed individually.
\item \emph{An algorithm to generate calls to BLAS routines}.
\item \emph{Autotuning}. We implement a model-driven, dynamic autotuner that transparently evaluates multiple sets of code transformations to determine the best optimization strategy for a given PDE. The main challenge here is to build, for a generic problem, a reasonably small search space that comprises most of the effective code variants.
\end{itemize} 

Expression rewriting and code specialization have been implemented in COFFEE, which in turn is integrated with the Firedrake framework. Therefore, to testify the goodness of our approach, we provide an extensive and unprecedented performance evaluation across a number of forms of increasing complexity, including some based on complex hyperelasticity models. We characterize our problems by varying polynomial order of the employed function spaces and number of pre-multiplying functions. To clearly distinguish the improvement achieved by this work, we will compare, for each test case, four sets of code variants: 1) unoptimized code, i.e. a local assembly routine as returned from the form compiler; 2) code optimized by FEniCS, i.e. the work in~\cite{quadrature1}; 3) code optimized as described in~\cite{Luporini}; code optimized by expression rewriting and code specialization as described in this paper. Notable performance improvements of 4) over 1), 2) and 3) are reported and discussed.



\section{Preliminaries}
\subsection{Overview of the Finite Element Method}
\subsection{From Math to Actual Code}

\section{On the Space of Possible Code Optimizations}
\subsection{Outline of COFFEE}

\section{Expression Rewriting}

\section{Code Specialization}

\section{Composing Optimizations}

\section{Design and Implementation of COFFEE}
\subsection{Input ad Output: the Integration with Firedrake}
\subsection{Structure}
\subsection{Conveying Domain-Specific Knowledge}

\section{Performance Analysis}
\subsection{Contribution of Basic Optimizations}
\subsection{Evaluation in Forms of Increasing Complexity}

\section{Generalizing to Local Vector Assembly}

\section{Conclusion}